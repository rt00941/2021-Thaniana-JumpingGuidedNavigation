% Document class
\documentclass[fontsize=12pt, paper=a4, headinclude, twoside=false, parskip=half+, pagesize=auto, numbers=noenddot, plainheadsepline, open=right, toc=listof]{scrreprt}
\usepackage[standardsections]{scrhack}

% PDF compression
\pdfminorversion=5
\pdfobjcompresslevel=1

% General
%\usepackage[automark]{scrpage2}          % Header and Footer
\usepackage[automark]{scrlayer-scrpage}  % Header and Footer
\usepackage{amsmath, marvosym}           % Math libraries
\usepackage[utf8]{inputenc}              % UTF8-Encoding
\usepackage{amsthm}
\usepackage[bottom]{footmisc}
\usepackage[acronym]{glossaries}
\usepackage{pdfpages}

% Font
\usepackage{mathpazo}                    % Palatino for math
\usepackage{setspace}                    % Line space
\onehalfspacing                          % Line space 1.5
\setkomafont{chapter}{\Huge\rmfamily}    
\setkomafont{section}{\Large\rmfamily}
\setkomafont{subsection}{\large\rmfamily}
\setkomafont{subsubsection}{\large\rmfamily}
\setkomafont{chapterentry}{\large\rmfamily}
\setkomafont{descriptionlabel}{\bfseries\rmfamily}
\setkomafont{captionlabel}{\small\bfseries}
\setkomafont{caption}{\small}
\usepackage{amssymb}
\usepackage{tcolorbox}

% Language: English
\usepackage[british]{babel}

% PDF-Output
\usepackage[british, colorlinks=false, linkcolor=black]{hyperref}
\usepackage[final]{microtype}
\usepackage{url}
\usepackage{pdflscape}

% Tables
\usepackage{multirow}
\usepackage{multicol}
\usepackage{tabularx}
\usepackage{array}
\usepackage{float}
\usepackage{varwidth}
\usepackage{verbatimbox}

% Images
\usepackage{graphicx}
\usepackage{color}
\graphicspath{{images/}}
\DeclareGraphicsExtensions{.pdf,.png,.jpg}
\usepackage{subfigure}
\newcommand{\subfigureautorefname}{\figurename}
\usepackage[all]{hypcap}
\setcapindent{0em}
\setcapwidth[c]{0.9\textwidth}
\setlength{\abovecaptionskip}{0.2cm}

% Source code
\usepackage{listings}

\definecolor{mygreen}{rgb}{0,0.6,0}
\definecolor{mygray}{rgb}{0.5,0.5,0.5}
\definecolor{mymauve}{rgb}{0.58,0,0.82}

\lstset{ %
  backgroundcolor=\color{white},   % choose the background color
  basicstyle=\scriptsize\ttfamily,        % size of fonts used for the code
  breaklines=tduringrue,                 % automatic line breaking only at whitespace
  captionpos=b,                    % sets the caption-position to bottom
  commentstyle=\color{mygreen},    % comment style
  escapeinside={\%*}{*)},          % if you want to add LaTeX within your code
  keywordstyle=\color{blue},       % keyword style
  stringstyle=\color{mymauve},     % string literal style
  frame=single
}

% Quotes
\usepackage{epigraph}
\setlength\epigraphwidth{10cm}
\setlength\epigraphrule{0pt}
\usepackage{etoolbox}
\makeatletter
\patchcmd{\epigraph}{\@epitext{#1}}{\itshape\@epitext{#1}}{}{}
\makeatother

% Title style
\usepackage{titlesec, blindtext}
\definecolor{gray75}{gray}{0.75}
\newcommand{\hsp}{\hspace{20pt}}
\titleformat{\chapter}[block]{\Huge\bfseries}{\thechapter\hsp\textcolor{gray75}{$\vert$}\hsp}{0pt}{\Huge\bfseries}

% Title spacing
\titlespacing*{\chapter}{0pt}{0pt}{20pt}


% left-aligned footnotes
\deffootnote{1.5em}{1em}{\makebox[1.5em][l]{\thefootnotemark}}

\typearea{14}
\makeatletter
\newcommand{\saved@equation}{}
\let\saved@equation\equation
\def\equation{\@hyper@itemfalse\saved@equation}
\makeatother

% cites with text in them
\makeatletter
\let\cite\relax
\DeclareRobustCommand{\cite}{%
  \let\new@cite@pre\@gobble
  \@ifnextchar[\new@cite{\@citex[]}}
\def\new@cite[#1]{\@ifnextchar[{\new@citea{#1}}{\@citex[#1]}}
\def\new@citea#1{\def\new@cite@pre{#1}\@citex}
\def\@cite#1#2{[{\new@cite@pre\space#1\if\relax\detokenize{#2}\relax\else, #2\fi}]}
\makeatother

% Hypothesis Counters
% Parameters of newtheoremstyle: ABOVESPACE, BELOWSPACE, BODYFONT, INDENT, HEADFONT, HEADPUNCT, HEADSPACE, CUSTOM-HEAD-SPEC

\newtheoremstyle{hypothesisstyle}  
  {\topsep}
  {0}
  {\normalfont}
  {0pt}
  {\bfseries}
  {:}
  {5pt plus 1pt minus 1pt}
  {\thmname{#1}\textsubscript{\thmnumber{#2}}}

\newtheoremstyle{shypothesisstyle}
  {\topsep}
  {0}
  {\small}
  {5pt}
  {\itshape\bfseries}
  {:}
  {5pt plus 1pt minus 1pt}
  {\thmname{#1}\textsubscript{\thmnumber{#2}}}

\theoremstyle{hypothesisstyle}
\newtheorem{hypothesis}{H}[]
\theoremstyle{shypothesisstyle}
\newtheorem{shypothesis}{H}[hypothesis]
\renewcommand*{\theshypothesis}{\arabic{hypothesis}\alph{shypothesis}}
\usepackage[capitalise]{cleveref}
\crefformat{hypothesis}{\textbf{H\textsubscript{#1}}}
\crefformat{shypothesis}{\textbf{\textit{H\textsubscript{#1}}}}
\newcommand{\hypref}[1]{\hyperref[#1]{\cref{#1}}}

% Research Question Counters
% Parameters of newtheoremstyle: ABOVESPACE, BELOWSPACE, BODYFONT, INDENT, HEADFONT, HEADPUNCT, HEADSPACE, CUSTOM-HEAD-SPEC

\newtheoremstyle{rqstyle}  
{\topsep}
{0}
{\normalfont}
{0pt}
{\bfseries}
{:}
{5pt plus 1pt minus 1pt}
{\thmname{#1}\textsubscript{\thmnumber{#2}}}

\newtheoremstyle{srqstyle}
{\topsep}
{0}
{\small}
{5pt}
{\itshape\bfseries}
{:}
{5pt plus 1pt minus 1pt}
{\thmname{#1}\textsubscript{\thmnumber{#2}}}

\theoremstyle{rqstyle}
\newtheorem{researchq}{RQ}[]
\theoremstyle{srqstyle}
\newtheorem{sresearchq}{RQ}[researchq]
\renewcommand*{\thesresearchq}{\arabic{researchq}\alph{sresearchq}}
\usepackage{cleveref}
\crefformat{researchq}{\textbf{RQ\textsubscript{#1}}}
\crefformat{sresearchq}{\textbf{\textit{RQ\textsubscript{#1}}}}
\newcommand{\researchqref}[1]{\hyperref[#1]{\cref{#1}}}


% further commands
\newcommand{\todo}[1]{\colorbox{red}{\textbf{TODO:} #1}}
\newcommand{\degree}[1]{#1^{\circ}}

\providepairofpagestyles{myThesis}{%
	\clearpairofpagestyles
	\cfoot*{\pagemark}
}

% acronyms 
\newacronym{vr}{VR}{Virtual Reality}
\newacronym{hmd}{HMD}{Head-Mounted Displays}
\newacronym{ve}{VE}{Virtual Environments}
\newacronym{3d}{3D}{3 Dimensonal}
\newacronym{ui}{UI}{User Interfaces}
\newacronym{rtlx}{RTLX}{Raw Task Load Index}
\newacronym{prs}{PRS}{Path Recall Score}

% buw title page
\makeatletter
\newcommand*{\studyprogramme}[1]{\def\@studyprogramme{#1}}
\newcommand*{\matnr}[1]{\def\@matnr{#1}}
\newcommand*{\birthdate}[1]{\def\@birthdate{#1}}
\newcommand*{\birthplace}[1]{\def\@birthplace{#1}}
\newcommand*{\firstref}[1]{\def\@firstref{#1}}
\newcommand*{\secondref}[1]{\def\@secondref{#1}}
\newcommand*{\submissiondate}[1]{\def\@submissiondate{#1}}

\def\maketitle
{%
\clearscrheadings
\clearscrplain

Bauhaus-Universität Weimar \\
Faculty of Media \\
Degree Programme \@studyprogramme

\vspace{30mm}

\begin{center}

\begin{huge}
\@title
\end{huge}\\ \vspace{0.25cm}
\begin{LARGE}
\@subtitle
\end{LARGE}


\vspace{13mm}
\begin{Large}
Master's Thesis
\end{Large}
\end{center}

\vspace{25mm}

\@author \hfill Matriculation number: \@matnr \\
born on \@birthdate{} in \@birthplace

\vspace{3.5cm}

\begin{tabular}{@{}ll}
{\textbf{First referee:}} & \@firstref \\
{\textbf{Second referee:}} & \@secondref \\
\end{tabular}

\vspace{1cm}
Submission date: \@submissiondate

\clearpage
}
\makeatother