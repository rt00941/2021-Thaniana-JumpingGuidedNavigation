\label{Chapter:Design of the User Study}
\section{Research Questions}
\label{section:DUS Research Questions}
In Chapter \ref{Chapter:Automated Guided Jumping for Navigation} we looked at a technique for automated guided navigation using the jumping metaphor and in Chapter \ref{Chapter:Comprehensibility of Jumps} we saw how the jumps in this technique could be made comprehensible so that the user would know when and where they will jump. The motivations and scenarios that might require such a technique were discussed in Chapter \ref{chapter:Guided Jumping Scenarios}. Keeping in mind the motivation to have a virtual museum that novice \acrshort{vr} users are able to explore, we introduced the following research questions:

\begin{researchq}
	\label{rq:rq1}
	How can guided navigation techniques facilitate the acquisition of relevant knowledge of the scene while avoiding motion sickness?
\end{researchq}
\begin{researchq}
	\label{rq:rq2}
	How can we maximize the comprehensibility of a sequence of automated jumps?
\end{researchq}
\begin{researchq}
	\label{rq:rq3}
	Will having guided jumping improve comfort and reduce task load compared to free jumping with visual guidance?
\end{researchq}
%Research question \cref{rq:rq1} is very importance and must be referenced in the text.
To study the developed technique with regards to these research questions we decided to design a study that would compare automated jumping with user controlled (free) jumping. It was important to have a controlled study designs such that there would be no other influencing variables besides the automation.