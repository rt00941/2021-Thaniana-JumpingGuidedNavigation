\label{Chapter:Guided Jumping Motivation}
In the previous chapter we discussed navigation techniques and why guiding for interaction to show why we decided to design a guided navigation technique using jumping instead of steering. In this chapter we will look at use cases where navigation in a \acrshort{ve} is required to further demonstrate the motivation for automated guided jumping for navigation. We will also introduce our research questions for this thesis. 

Guiding facilitates exploration of \acrshort{3d} data where it is \textit{'arranged on purpose'} and also applications in which \textit{'the structure and meaning of the 3D data is unknown'}. Beckhaus focused on the former types of applications and tested their CubicalPath system for guided exploration on such applications. This included a Virtual Art Museum~\cite{Beckhaus2002}. These types of applications can include those where a narrative structure needs to be provided within a \acrshort{vr} experience~\cite{Galyean1995} or where complete environmental knowledge is essential~\cite{Freitag2018}. This particularly lends itself to Storytelling in \acrshort{vr} and Virtual Tours. 

\section{Storytelling in VR}
\label{section:GJM Storytelling in VR}
Storytelling, the art of sharing stories has existed in humanity for millennia. As society has developed so have the ways in which storytelling is accomplished. With the advent of a technological age this storytelling transformed onto screens. Now, in recent times there has been a further breakthrough in storytelling with the advances in \acrshort{vr} and the increase in ways of presenting immersive content. As Bucher explains the concept of \acrshort{vr} has existed long before the technology itself yet storytelling in \acrshort{vr} follows different rules from the traditional stories as the perspective of a story is different in an Immersive Environment~\cite{Bucher2017}.

This makes it quite compelling to look into \acrshort{vr} techniques that could be used to support this crafting of immersive narratives. Quite a few techniques for designing better \acrshort{ve}s for storytelling are already exist but what is the best way of allowing users to move around in these \acrshort{ve}s? We need to ensure that a structure is provided instead of just allowing an \textit{emergent narrative}, which is a story that emerges as a \textit{'product of our interactions and goals as we navigate the experience'}. This is where guiding would come in to \textit{'balance the interaction (exploration) with an ability to guide the user, while at the same time maintaining a sense of pacing or flow through the experience'}~\cite{Galyean1995}.

According to Rodriguez et, al. 



\section{Virtual Tours}
\label{section:GJM Virtual Tours}

\section{Research Questions}
\label{section:GJM Research Questions}

\begin{researchq}
	\label{rq:rq1}
	How can guided navigation techniques facilitate the acquisition of relevant knowledge of the scene while avoiding motion sickness?
\end{researchq}
\begin{researchq}
	\label{rq:rq2}
	How can we maximize the comprehensibility of a sequence of automated jumps?
\end{researchq}
\begin{researchq}
	\label{rq:rq3}
	Will having guided jumping improve comfort and reduce task load compared to free jumping with visual guidance?
\end{researchq}

\section{Conclusion}
\label{section:GJM Conclusion}
