\label{chapter:Related Work}
%This is related work with an example citation~\cite{Galyean1995}...
As mentioned in Chapter \ref{chapter:Introduction}, this thesis aims to investigate a technique for automated guided navigation using the jumping metaphor. To understand where the concept for this technique comes from we will take a look at different navigation metaphors for \acrshort{hmd}s and see what the advantages of jumping navigation are. We will then see what the purpose of guiding in \acrshort{ve}s is and why it can be useful for navigation to be guided. Based on this we will then show the motivation for bringing together jumping and guiding into one navigation technique. 

\section{Navigation}
\label{section:RW Navigation}

Navigation is the task of moving around and when it comes to \acrfull{3d} environments it is one of the most common actions that is carried out by users. According to Bowman et al. navigation \textit{"presents challenges such as supporting spatial awareness, providing efficient and comfortable movement between distant locations, and making navigation lightweight so that users can focus on more-important tasks"}. Navigation can be divided into the motor and cognitive components, travel and way finding respectively. Navigation tasks include exploration, search and maneuvering. \cite{Bowman2001}. Our technique will focus on 
\begin{itemize}
	\item Exploration: Navigation with no explicit target for the purpose of investigating the environment.
	\item Search: Navigation with the intention of going to a target which is known or finding one which is not known.
\end{itemize}

\subsection{Navigation Challenges}
\label{subsection RW Navigation:Navigation Challenges}
The navigation challenges outlined by Bowman et al. were:
\begin{itemize}
	\item Supporting spatial awareness.
	\item Providing efficient and comfortable movement between distant locations.
	\item Making navigation lightweight so that users can focus on more-important tasks.
\end{itemize}
In addition to these challenges there is one additional challenge when navigating using \acrshort{hmd}s:
\begin{itemize}
	\item Reducing motion sickness. 
\end{itemize}

\subsubsection{Supporting Spatial Awareness}
\label{subsubsection RW Navigation NC: Supporting Spatial Awareness}
Spatial awareness is defined as \textit{"the user’s implicit knowledge of his position and orientation within the environment during and after travel"} by Bowman et al. \cite{Bowman1997}.

motion sickness and spatial awareness
steering vs teleportation
introduce jumping 
user controlled versus automatic navigation


\section{Guiding}
\label{section:RW Guiding}
guiding in vr definition
motivations for guiding
guiding examples
river analogy, exploration assistance etc
guiding for navigation
user controlled versus automatic guiding

\section{Conclusion}
\label{section:RW Conclusion}
combine navigation and guiding to get guided navigation using jumping. then we go to motivation