\label{Chapter:Related Work}
As mentioned in \cref{Chapter:Introduction}, this thesis aims to investigate a technique for automated guided navigation using the jumping metaphor. To understand where the concept for this technique comes from, we will take a look at different navigation metaphors for \acrshort{hmd}s and see what the advantages of jumping navigation are. We will then see what the purpose of guiding in \acrshort{ve}s is and why it can be useful for navigation to be guided. Based on this, we will then show the motivation for bringing together jumping and guiding into one navigation technique. 

\section{Navigation}
\label{section RW: Navigation}

Navigation is the task of moving around and when it comes to \acrfull{3d} environments it is one of the most common actions that is carried out by users. According to Bowman, Kruijff et al. navigation \textit{'presents challenges such as supporting spatial awareness, providing efficient and comfortable movement between distant locations, and making navigation lightweight so that users can focus on more-important tasks'}. Navigation can be divided into the motor and cognitive components, travel, and way-finding, respectively. Navigation tasks include exploration, search, and manoeuvring~\cite{Bowman2001}. Our technique will focus on exploration, which is navigation with no explicit target to investigate the environment and search, which is navigation to go to a target that is known or finding one which is not known.

\subsection{Quality Factors}
\label{subsection RW Navigation: Quality Factors}
Quality factors of a technique are what makes it effective. Some of the quality factors that we decided should be taken into consideration before designing and comparing navigation techniques for Immersive \acrshort{ve}s from the quality factors outlined by Bowman, Koller et al. are as follows:
\begin{enumerate}
	\item \textit{'Spatial Awareness'}, which is a user's knowledge of their position and orientation in an environment while travelling. We want to focus on this because we want our technique to be useful for navigating spaces where the environment is important and therefore, we do not want our technique to compromise on spatial awareness.
	\item \textit{'Ease of Learning}, which is how easy the technique is for a novice user to learn. This quality factor is important because we want our technique to be useful for novice users so that they do not need to spend too much time learning the technique and can easily understand it.
	\item \textit{'Ease of Use'}, which is how complex a user finds using the technique.~\cite{Bowman1997}. This is again relevant for our technique as we want users to be able to focus on the environment or task rather than have task load from using the technique. 
	
	In addition to the above, there is also a final quality factor:
	\item \textit{'Feeling Well'}, which is that the technique should minimize simulator sickness, as it can be a great concern for some users in \acrshort{ve}s~\cite{LaViola2000}. Reducing simulator sickness is an essential part of our design decisions as we want users to be comfortable using our technique rather than wanting to stop.
\end{enumerate}

\subsection{Travel Metaphors}
\label{subsection RW Navigation: Travel Metaphors}
Travel metaphors can be divided into many categories such as physical movement, viewpoint manipulation, steering, target-based and route planning~\cite{Bowman2001}. These different metaphors consider the quality factors as mentioned in \cref{subsection RW Navigation: Quality Factors} to different extents depending on the goal of navigation. When the goal is not to travel but some other task which requires navigation, the technique must be more simplistic to not take away focus from the task. Hence, two very common metaphors of travel used in such cases are steering and teleportation, which is a form of target based travel.

Steering navigation is a continuous movement of the user's viewpoint in a specific direction that is controlled by gaze, pointing or a physical device. Velocity may also be varied as an additional action. When steering, the user is moving continuously through the surrounding environment and hence, has good spatial awareness. However, the movement may cause motion sickness as they are physically standing still while their surroundings are moving. ~\cite{Habgood2018}.

Teleportation, which is also a technique where users are physically standing still, tries to reduce motion sickness by moving directly to a target instead of continuously travelling through the surroundings. Habgood et al. mention that a solution to the problem of simulator sickness \textit{'has been to provide locomotion through teleportation'}~\cite{Habgood2018}. However, due to this discrete movement, users may miss out on parts of the environment as they go directly to another position and orientation than they were in. Therefore, when using teleportation, users would have less spatial awareness than if they were steering. Habgood et al. try to reduce this by proposing a \textit{'node-based navigation system which allows the player to move between predefined node positions using a rapid, continuous, linear motion'}~\cite{Habgood2018}.

Weissker et al. on the other hand try to reconcile steering and teleportation metaphors to minimize motion sickness while still getting a similar spatial awareness (or spatial updating) compared to steering, through the jumping metaphor. This is a short-range teleportation technique because it is target based travel to visible parts of the scene. Findings by Weissker et al. were that this technique resulted in \textit{'significantly faster travel times'} as compared to steering but \textit{'similar spatial updating accuracies in both conditions'} for $75\%$ of the participants. It also \textit{'induced significantly less simulator sickness'}. Therefore, the technique can be used in most cases as an alternative for steering. However, there may be some individuals that would have their ability for spatial updating impaired~\cite{Weissker2018}.

Based on this, we decided that we would like to use jumping navigation for our technique, moving users through sets of nodes at points of interest with way-points in between each node to be jumped to. Due to this, we felt that our technique would not work well for navigation tasks that related to manoeuvring, although they would be suited for tasks going through an environment to find a specific target (search) or to explore (exploration). 

\section{Guiding for Navigation}
\label{section RW: Guiding for Navigation}
According to Beckhaus, most common travel metaphors rely on direct user control~\cite[p. 6]{Beckhaus2002}. However, there have been studies that propose travel metaphors that are more automatic. Examples of these include:
\begin{itemize}
	\item The river analogy ~\cite{Galyean1995}.
	\item Navigation guidance to reduce cognitive load of way-finding~\cite{Elmqvist2005}~\cite{Elmqvist2008}.
	\item Managing coherent groups using path planning to dynamically move them~\cite{Silveira2008}.
	\item Dynamic potential fields for guided exploration or guided exploration through automated target based travel~\cite{Beckhaus2002}. 
\end{itemize}

Beckhaus also found that \textit{'self-navigation, automated travel, and guided exploration in a virtual environment'} worked together to form a \textit{'supportive system'} that provided users with helpful guided navigation without: confusing them, removing their self control or feeling of presence~\cite[p. 8]{Beckhaus2002}. This shows that a balance of user control and automation is useful. This also introduces us to the concept of guided exploration.

Guiding means helping someone find a target object or location. Guiding through an environment would mean showing them the recommended path through it to give them the best experience. This would still apply for a \acrshort{ve} but requires additional considerations that come with doing anything in \acrfull{vr}. Guiding can be potentially be done for any \acrshort{vr} interaction tasks, however, here we will talk about guiding for navigation in \acrshort{vr}.

The wish to \textit{'balance the notion of interaction with guidance (telling)'} and to provide a narrative structure to an experience motivates guided navigation techniques. Galyean presented the River Analogy which is an automatic steering technique for guided navigation and guides \textit{'the user’s continuous and direct input within both space and time allowing a more narrative presentation'}. The technique was useful when applied to a VR experience in a museum, showing that there are cases where guided interaction for \acrshort{ve}s is useful~\cite{Galyean1995}.

Another example of guided navigation presented by Freitag et al. uses more passive guiding and simply supplements a user's free exploration of the \acrshort{ve}. It visualizes the paths a user can follow based on what has already been explored and shows the users what their final location would be if they follow it. A study of this technique showed that it \textit{'improves the knowledge of the scene, leads to a more complete exploration, and is experienced as helpful and easy to use'}. In comparison, during free exploration users \textit{'miss important parts, leading to incorrect or incomplete environment knowledge and a potential negative impact on performance in later tasks'}, thus showing the importance of guiding when complete environmental knowledge is essential or beneficial~\cite{Freitag2018}.

\section{Conclusion}
\label{section RW: Conclusion}
\cref{section RW: Navigation} and \cref{section RW: Guiding for Navigation} explored research on navigation techniques and why guiding can be useful for interaction in \acrshort{vr}, particularly navigation. Based on this research, we realized how useful a guiding technique using jumping would be and the considerations that need to be taken when designing it. The motivations for this will be discussed further in \cref{Chapter:Guided Jumping Motivation} and the considerations when designing it will be further discussed in \cref{Chapter:Automated Guided Jumping}.