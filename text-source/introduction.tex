\label{Chapter:Introduction}
Many navigation techniques exist for both Desktop and Immersive \acrfull{ve} that define how users move around these \acrshort{ve}s. The goals of navigation are to move towards a target location and orientation to explore the environment. Navigation should facilitate way-finding in the \acrshort{ve}, which means allowing the user to know where they are, where they will go next and how they will get there. This also means that the user should have a good perception of the \acrshort{ve} and path that they took. Navigation techniques have to ensure that there is minimal motion sickness; sufficient environmental awareness which means that while navigating the user knows where they are in an environment compared to where they were before; and that it is easy to reach important places in the environment. Two common metaphors for navigation are steering and teleportation.

Steering navigation is a technique where there is continuous movement in a direction indicated either by gaze, pointing, or use of a physical device. In some cases an additional action can be added to specify the velocity. With steering navigation, spatial awareness is generally good but can cause motion sickness. Teleportation navigation is a target-based metaphor that discretely specifies where the goal position is by pointing or choosing a location and orientation to be moved towards. This form of navigation minimizes motion sickness but results in less environmental awareness as compared to the steering metaphor. Some techniques try to reconcile these two metaphors to minimize motion sickness while still maintaining a good environmental awareness. One example is the jumping metaphor presented by Weissker et al. which \textit{'only allows to teleport to locations in the currently visible part of the scene'} which makes it a short-range version of the teleportation metaphor~\cite{Weissker2018}. 

Navigation techniques can be active such that the user is controlling their movement; passive such that the user is being automatically moved around the environment; or they can be a mix of active and passive. Navigation techniques can also provide the user with guidance, regardless of whether this is active or passive. The river analogy presented by Galyean, which guides \textit{'the user’s continuous and direct input within both space and time allowing a more narrative presentation'}, using automatic steering for guided navigation is a guided navigation technique. These allow for the addition of a narrative structure to a \acrshort{ve}~\cite{Galyean1995}. In this work, we will explore guided navigation using the jumping metaphor instead of a steering one and investigate the benefits of an automatic approach over a user-controlled one for a museum setting. 

This thesis will discuss work related to navigation techniques and guiding in \acrshort{ve}s on \acrfull{hmd}s in \cref{Chapter:Related Work}. Then in \cref{Chapter:Guided Jumping Motivation} it will look into the motivation and use cases for a technique that combines guiding with navigation for an automated guided navigation technique using jumping. Based on these motivations, \cref{Chapter:Automated Guided Jumping} will look at the design and development of an automated guided navigation technique. To justify the benefits of this proposed technique in comparison to a technique with free jumping using visual guidance, a study design and procedure will be introduced in \cref{Chapter:Design and Procedure of the User Study}. Then \cref{Chapter:Evaluation of the User Study} will look at the results of this study. \cref{Chapter:Conclusion and Future Work} will then conclude the thesis by summarizing the results and propose some future work that can be done on the topic of automated guided jumping for navigation in \acrshort{ve}s.

All in all, the contributions of this thesis are as follows:
\begin{itemize}
	\item Discussion of related work on navigation techniques, including quality factors, travel metaphors and guiding techniques for navigation.
	\item Exploration of use cases that motivate the need for an automated guided jumping navigation technique.
	\item Implementation of a potential interaction design for automated guided jumping, including the way jumps take place and travel feedback.
	\item Presentation of the design, procedure, and evaluation of a user study with 9 participants comparing our automated guided jumping technique with a free jumping technique using visual guidance.
\end{itemize}
