\label{Chapter:Conclusion and Future Work}
In this thesis we looked into the need for a guided jumping navigation technique for immersive virtual environments.

We started with a look at related work in navigation and guiding for navigation. We looked into three travel metaphors, steering, teleportation and jumping. This led to a decision to focus on the jumping metaphor as it balances a good spatial understanding with reduced simulator sickness for a majority of users. We also looked into guiding navigation techniques that included techniques with automatic guiding using steering navigation and a free guiding with exploration assistance. We decided to focus on an automatic guiding technique using jumping navigation and compare it with a technique using free jumping with visual guidance. The only difference between these techniques would be the level of user control to jump. We decided on developing an automated guided jumping technique because wanted a technique that would provide users with the ability to navigate a completely new environment with a pre-existing narrative structure without the need for tour guides to guide them through the environment. We also wanted the interface to be novice friendly and allow users to focus on the environment rather than the navigation technique. Finally, we wanted the technique to allow users to get relevant knowledge of the environment and avoid simulator sickness or any other kind of discomfort. We also noted that the such an automated guided jumping technique could be quite useful in environments where a narrative structure is important such as for virtual tours or storytelling in \acrshort{vr}.

After a look at the literature review and motivation we looked into the design and development of the automated guided jumping technique. The technique was developed such that the user would be automatically jumping from one node or way point to another. Nodes in this case are points of interest while way points are points in between the nodes to break the navigation from one node to the next into smaller distances such that it could be called a jump and users would not miss anything. To ensure users were not surprised by the jumps we also designed and developed visual feedback about the jump including avatars at the next position and orientation, a visual countdown before a jump and signs for pausing and making a choice. Finally, to allow users to have some control over pausing, resuming and making a choice we implemented simple hand gestures that users could use.

To explore the benefits of our technique we decided to conduct a study comparing it with a technique that would allow users to jump freely instead of being automatically guided. This free jumping technique would also have visual guidance that would be the same as the visual feedback given to the users in our automated guiding technique. The study comparing these techniques was conducted using 9 participants. The results from this study show