\label{Chapter:Conclusion and Future Work}
In this thesis we looked into the need for a guided jumping navigation technique for immersive virtual environments.

We started with a look at related work in navigation and guiding for navigation. We looked into three travel metaphors, steering, teleportation and jumping. This led to a decision to focus on the jumping metaphor as it balances a good spatial understanding with reduced simulator sickness for a majority of users. We also looked into guiding navigation techniques that included techniques with automatic guiding using steering navigation and a free guiding with exploration assistance. We decided to focus on an automatic guiding technique using jumping navigation and compare it with a technique using free jumping with visual guidance. The only difference between these techniques would be the level of user control to jump. We decided on developing an automated guided jumping technique because wanted a technique that would provide users with the ability to navigate a completely new environment with a pre-existing narrative structure without the need for tour guides to guide them through the environment. We also wanted the interface to be novice friendly and allow users to focus on the environment rather than the navigation technique. Finally, we wanted the technique to allow users to get relevant knowledge of the environment and avoid simulator sickness or any other kind of discomfort. We also noted that the such an automated guided jumping technique could be quite useful in environments where a narrative structure is important such as for virtual tours or storytelling in \acrshort{vr}.

After a look at the literature review and motivation we looked into the design and development of the automated guided jumping technique. The technique was developed such that the user would be automatically jumping from one node or way point to another. Nodes in this case are points of interest while way points are points in between the nodes to break the navigation from one node to the next into smaller distances such that it could be called a jump and users would not miss anything. To ensure users were not surprised by the jumps we also designed and developed visual feedback about the jump including avatars at the next position and orientation, a visual countdown before a jump and signs for pausing and making a choice. Finally, to allow users to have some control over pausing, resuming and making a choice we implemented simple hand gestures that users could use.

To explore the benefits of our technique we decided to conduct a study comparing it with a technique that would allow users to jump freely instead of being automatically guided. This free jumping technique would also have visual guidance that would be the same as the visual feedback given to the users in our automated guiding technique. The study comparing these techniques was conducted using 9 participants. The results from this study show that while users had mostly positive things to say about the guided jumping navigation technique they still preferred free jumping because of the level of control they had with it, indicating that the automated jumping technique needs to be refined in a way that users feel more comfortable with not having control. The results also indicate that in general automated guided navigation does facilitate acquisition of relevant knowledge of the environment and avoids motion sickness even though motion sickness levels were higher than with free jumping. 

Through the study we also realized that comprehensibility of jumps was not the best with our technique but could be improved as future work. Some ways in which to improve it is to refine the gesture control further by a deeper look at the hand tracking functionalities for the \acrshort{hmd}s. Perhaps different \acrshort{hmd}s could be compared for their hand tracking to determine which one gives the best user experience. Another improvement could be to the visual countdown for the time. This could be made more visible or another way of visualizing it could be used instead of the line that goes narrower as has been used in the current implementation. Finally, the avatar that indicates the next position could be further improved so that the next position shown is even clearer. 

The study also showed us that while the comfort and task load scores were good for the automated guided jumping technique they were not better than for free jumping with visual guidance. This means some further studies need to be conducted to find ways in which the automated guided jumping technique can be improved so that the comfort and task load for it could be equal to or better than for free jumping with visual guidance.  As we can see from our findings there are definitely more open questions that could not fit into the scope of this thesis that can be looked at in future work on automated guided jumping to improve on the implementation suggested in this thesis.

One such question is the narrative structure in the environment. The current implementation only placed nodes at points of interests and did not focus on what these could provide to the experience. There is potential to explore how the environment can be set up to have an interesting narrative structure and where time and other factors could be part of the design space of an environment that would use an automated guided navigation technique. In particular, implementations of such a technique for a storytelling use case could be further explored because there the focus of users is more on the storytelling and less on the navigation.

Another question to be asked is how the technique could be extended for multi user cases such that groups of users can be guided through the environment together. In such a case there would be additional questions related to group navigation that would arise.

Future work can also look into ways that the narrative structure could be created dynamically during or before the exploration either by one user in a group of users or just a single user rather than being pre loaded into the experience. There are a lot of possible scenarios for which implementations where experiences are created dynamically and can be considered, for example, an experience using automated guiding navigation may be set up in a museum which allows users to have a custom tour. In this case there would need to be a way to customize the tour. Another example could be that there is a group of users in which one user is setting the path and the others are being moved using automated guided jumping.

All in all, automated guided jumping has its benefits and use cases in which it can be very beneficial and therefore, there is room to explore different implementations for it. While we have presented one implementation that focuses on the jumps and the travel feedback there is still the challenge to improve travel feedback for more comfort and better comprehensiblity of jumps, as well as the challenges of designing better experiences, hardware and interaction techniques for automated guided jumping. There also remains the challenge of extending this into multi user \acrshort{vr}. Future work in this direction should ensure that users do not feel bothered by the lack of control of such a technique.

 